%-----------EXPERIENCE-----------
\section{Experience}
  \resumeSubHeadingListStart

    \resumeSubheading
      {Working Student Software Engineer }{Nov. 2024 -- Present}
      {\href{https://www.vector.com/}{Vector Informatik GmbH} - VSceneCreator (3D \href{https://www.asam.net/standards/detail/opendrive/}{OPENDrive} Editor)}{Munich, DE}
      \resumeItemListStart
        \resumeItem{Helped ship the 0.1 alpha version of the VSceneCreator (Unity/C\#) by suggesting, designing and implementing features, addressing bugs, and creating tests through close collaboration with Product Management.}
        \resumeItem{Architected a read-only "Explore Mode" with a hierarchical overview tree and dynamic tooltip system, segregating access to editing features resulting in a preview tool for licensed users and an improved UX for free users.}
        \resumeItem{Improved edit mode switching performance by restructuring the base model to enable lazy loading for non-essential and non-visible objects resulting in up to 70\% faster switch times.}
        \resumeItem{Optimized file loading through batched object creation and an enhanced progress bar replacing application freezes resulting in 10\% faster loading times, reduced memory usage and a better UX.}
        \resumeItem{Implemented support for single-sided lanes, enabling users to load OPENDrive files with single-sided lanes by automatically converting them to normal lanes resulting in prevention of broken files.}
        \resumeItem{Streamlined developer tools by improving Unity Rider integration resulting in 10\% faster asset refresh times.}
      \resumeItemListEnd
    
      \resumeSubheading
      {Working Student Software Developer}{Mar. 2021 -- Oct. 2024}
      {\href{https://www.vector.com/int/en/products/products-a-z/software/dyna4/}{Vector Informatik GmbH - DYNA4 (Simulation environment for virtual driving tests)}}{Munich, DE}
      \resumeItemListStart
        \resumeItem{Developed features, fixed critical bugs, and enhanced UI using Java and Eclipse RCP, contributing to 5 major product releases (5.0 to 9.0) in direct collaboration with the Product Management.}
        \resumeItem{Engineered a custom view to assign aliases for trace signals with a new file format, a custom persistence layer and reducing expected user knowledge with errors and warnings via built-in validation and quickfixes.}
        \resumeItem{Enhanced usability by completely reworking the trace signal view, leading to a more modern and efficient UX.}
        \resumeItem{Improved user workflow and simulation project integrity by developing a utility dialog to find and remove unreferenced files and by refactoring a dialog to transparently display all unsaved artifacts.}
        \resumeItem{Reduced manual QA effort and accelerated development cycles by expanding automated test suites using SWTBot and JUnit in addition to enhancing the CI pipeline to link Jenkins results with JIRA tickets.}
        \resumeItem{Improved codebase maintainability and enabled faster subsequent development by refactorings of legacy systems.}
      \resumeItemListEnd

  \resumeSubHeadingListEnd
